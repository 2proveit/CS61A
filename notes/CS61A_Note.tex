\documentclass{article}
\usepackage{ctex}
\usepackage{graphicx}
\title{CS61A Learning Note}
\author{Seinloong Lee}
\date{2022.9.28}



\begin{document}
\maketitle
\begin{center}




\textbf{in memory of my CS61A learning time}
\end{center}

\newpage
\section{Lecture 0: Set up}
python installing and windowspowershell using
\par

\newpage
\section{Lecture 2: Functions}
\subsection{Type of expressions}
An expression describes a computation and evaluates to a value
\subsection{Call Expression in Python}
\subsection{Anatomy of a Call expression}
operators and oprands are also expressions
so they evaluate to values

1.Evaluate the operator and then the operand subexpressions
\par
2.Apply the function that is value of the operatoe subexpression to the arguments that are the values of the operand subexpression
\par
eg:nul(add(2,mul(4,6)),add(3,5))
\par
\subsection{Name, Assignmentt, and User=Defined Functions}
>>>from math import pi \par
>>>pi\par
>>>from math import sin\par
>>>radius = 10\par
>>>radius\par
>>>2*radius\par
>>>area, circ = pi*radius*radius, 2*pi*radius\par


\end{document}
