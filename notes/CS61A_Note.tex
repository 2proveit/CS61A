\documentclass{article}

\usepackage{graphicx}
\title{CS61A Learning Note}
\author{Seinloong Lee\thanks{written in \LaTeX}}
\date{2022.9.28}



\begin{document}

\begin{figure}
\centering
\includegraphics[trim=15 0 5 0,clip=noclip,scale=0.1]{cs61a}
\end{figure}
\maketitle
\begin{center}
\textbf{in memory of my CS61A learning time}
\end{center}

\newpage
\section{Lecture 1: Set up}
python installing and  windowspowershell using
\par

\newpage
\section{Lecture 2: Functions}
\subsection{Type of expressions}
An expression describes a computation and evaluates to a value
\subsection{Call Expression in Python}
\subsection{Anatomy of a Call expression}
operators and oprands are also expressions
so they evaluate to values

1.Evaluate the operator and then the operand subexpressions
\par
2.Apply the function that is value of the operatoe subexpression to the arguments that are the values of the operand subexpression
\par
eg:nul(add(2,mul(4,6)),add(3,5))
\par
\subsection{Name, Assignment, and User=Defined Functions}
>>>from math import pi \par
>>>pi\par
>>>from math import sin\par
>>>radius = 10\par
>>>radius\par
>>>2*radius\par
>>>area, circ = pi*radius*radius, 2*pi*radius\par
\subsection{Environment Diagrams}
Environment diagrams visualize the interpreter's process.\par
\subsection{Defining Functions}
Assigning is a simple means of abstrsction:binds means to values\par
Functoin definition is a more powerful means of abstraction : binds names to \emph{expression.}\par
def <name> (<format parameters>):\par
	return <return expressions>\par
	
\subsection{Calling User-Defined Functions}
Procedure for calling/applying user=defined functions(version 1):\par
1.Add a local frame, forming a \emph{new} environment. \par
2.Bind the function's formal parameters to its arguments in that frame.\par
3.Execute the body of the function in that new environment.\par
\subsection{Looking Up Names In environments}
Every expression is evaluated in the context of an environment.\par
So far, the current environment is either:\par
. The global frame alone,or\par
. A local frame, followed by the global frame.\par
\emph{Most impor two things I'll say all day:}\par
An environment is a sequence of frames.\par

A name evaluates to the value bound to that name in the earliest frame of the current environment in which that name is found.\par
\newpage
\section{Control}
\subsection{Print and None}
>>>print(-2)\par 
>>>None \par 
\subsection{None Indicates that Nothing has been returned}
The special value None represents nothing in Python.\par 
A function that does not explicitly return a value will return None.\par 
\emph{Careful:}None is not displayed by the interpreter as the value of an expression.\par 
\subsection{Pure Functions and Non-Pure functions}
Pure Functions return values\par 
Non-Pure Functions has side effects.\par 
A side effect isn't a value;it's anything that happens as a consequence of calling a function.\par 
\subsection{Nested Expression with Print}
\emph{the "return" function always returns a "None".}\par 
\subsection{Multiple environment}
Def statement: name parameters \par 
\emph{a new function is created! Name bound to that function in that current frame.}\par 
Call expression:




\end{document}
